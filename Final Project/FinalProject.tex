% Options for packages loaded elsewhere
\PassOptionsToPackage{unicode}{hyperref}
\PassOptionsToPackage{hyphens}{url}
%
\documentclass[
]{article}
\usepackage{amsmath,amssymb}
\usepackage{iftex}
\ifPDFTeX
  \usepackage[T1]{fontenc}
  \usepackage[utf8]{inputenc}
  \usepackage{textcomp} % provide euro and other symbols
\else % if luatex or xetex
  \usepackage{unicode-math} % this also loads fontspec
  \defaultfontfeatures{Scale=MatchLowercase}
  \defaultfontfeatures[\rmfamily]{Ligatures=TeX,Scale=1}
\fi
\usepackage{lmodern}
\ifPDFTeX\else
  % xetex/luatex font selection
\fi
% Use upquote if available, for straight quotes in verbatim environments
\IfFileExists{upquote.sty}{\usepackage{upquote}}{}
\IfFileExists{microtype.sty}{% use microtype if available
  \usepackage[]{microtype}
  \UseMicrotypeSet[protrusion]{basicmath} % disable protrusion for tt fonts
}{}
\makeatletter
\@ifundefined{KOMAClassName}{% if non-KOMA class
  \IfFileExists{parskip.sty}{%
    \usepackage{parskip}
  }{% else
    \setlength{\parindent}{0pt}
    \setlength{\parskip}{6pt plus 2pt minus 1pt}}
}{% if KOMA class
  \KOMAoptions{parskip=half}}
\makeatother
\usepackage{xcolor}
\setlength{\emergencystretch}{3em} % prevent overfull lines
\providecommand{\tightlist}{%
  \setlength{\itemsep}{0pt}\setlength{\parskip}{0pt}}
\setcounter{secnumdepth}{-\maxdimen} % remove section numbering
\ifLuaTeX
  \usepackage{selnolig}  % disable illegal ligatures
\fi
\IfFileExists{bookmark.sty}{\usepackage{bookmark}}{\usepackage{hyperref}}
\IfFileExists{xurl.sty}{\usepackage{xurl}}{} % add URL line breaks if available
\urlstyle{same}
\hypersetup{
  hidelinks,
  pdfcreator={LaTeX via pandoc}}

\author{}
\date{}

\begin{document}

 Final Project Proposal

\section{Project Context}\label{project-context}

\textbf{Through the eyes of Hardik Bishnoi :}

As the global landscape becomes increasingly interconnected,
international students often find themselves exploring new and diverse
experiences in unfamiliar territories. From my own experience, stepping
into foreign lands with limited resources commanded high-level resource
optimization to deal with cultural, geographical, and financial
challenges. One can say that such an experience is no different from a
hitchhiker trying to explore new lands with confined resources. Our
final project for algorithms is to model such a scenario with a
hitchhiker and understand how various variables affect optimal route
planning. I have always been interested in how various algorithms can
actually be put into practical path planning {[}1{]}. I want to utilize
the knowledge I have gained about various algorithms in class by
implementing this model and understand the nuances of implementation.

\textbf{Through the eyes of Rashaad Mirza:}

As an individual deeply passionate about the intersection of technology
and human experiences, I find profound personal relevance in our
project, which focuses on optimizing travel routes for hitchhikers.
Drawing parallels to the hitchhiker\textquotesingle s quest to explore
diverse places while minimizing costs, I see this project as a mirror to
the intricate dance of balancing constraints and optimizing outcomes.
This endeavor resonates with my curiosity for efficient systems and
aligns with my belief that technology can significantly enhance personal
experiences. It serves as an invaluable opportunity to delve into the
complexities of route optimization, akin to the hitchhiker strategically
navigating their path. Additionally, the implementation of concepts
learned in class and understanding their real-world applications excites
me, bridging the gap between theoretical knowledge and practical
solutions. Moreover, I am confident that the skills and insights gained
from this project will undoubtedly fortify my pursuit of another project
idea, focusing on the seamless and personalized integration of daily
tasks with travel experiences. Like every hitchhiker with a unique
story, I am enthusiastic about contributing to the narrative of our
project, weaving together the threads of exploration, efficiency, and
the human element in travel.

\textbf{Through the eyes of Saanidhya Vats:}

In our lives, particularly when we embark on a journey as tourists to a
new destination, our expectations for exploring the city can vary
significantly. Some individuals aspire to visit every renowned landmark
swiftly, while others prefer a more leisurely exploration of fewer
places. Some aim to navigate the city in a manner that minimizes overall
travel costs. Given that everyone is unfamiliar with the new
surroundings, pursuing specific objectives becomes challenging. Our
project seeks to address this challenge by offering diverse routes
tailored to individuals based on their available resources and specific
goals. This solution is not only beneficial for tourists but also
extends its utility to students, like ourselves, who are newcomers to a
city. Students often wish to discover the city while managing their
finances or earning some additional income. As a student, this project
holds significant relevance for us.

\textbf{Through the eyes of Hakshay Sundar:}

As a student, like many others, I too have a limited amount of resources
and the hope to explore more of the world around me. Thinking of which,
a hitch-hiker comes to mind, exploring at random the various places
around them and finding ways to maximize their resources. Doubling down
on that thought, I could relate to a few more processes such as: a CPU
scheduling tasks while taking into consideration a wide range of
constraints and working to optimize the output; animals moving from one
place to the other in the hopes of increasing their food supply,
potentially staying in a warmer climate, etc... This project allows me a
chance to try and understand these operations more closely to some
extent all while giving me a chance to model my hope of traveling around
and exploring more of this pale blue dot --- Pale Blue Dot by Carl
Sagan.

\section{Question}\label{question}

How would a hitchhiker on a road trip plan an optimal journey to manage
their resource?

\begin{itemize}
\item
  maze structure
\item
  algorithm
\item
  weight costs (random or selected)
\item
  different kinds of weights

  \begin{itemize}
  \item
    attractiveness of a node : whether the hitchhiker wants to stay at a
    node for longer etc
  \item
    mode of transport available including time constraint
  \end{itemize}
\item
  if hitchhiker can see the nodes or not
\item
  detect dead ends
\item
  end conditions:

  \begin{itemize}
  \item
    explore maximum nodes with minimum resources spent
  \end{itemize}
\item
  graph and document results
\end{itemize}

\section{Scope}\label{scope}

\section{Description}\label{description}

\textbf{References:}

{[}1{]} :
\href{https://developers.google.com/optimization/routing}{Vehicle
Routing, Google OR-Tools}

\end{document}
